\chapter{Introduction}
Illusions are created when our brains misinterpret the sensory information that it gathers. This is usually through optical and auditory illusions that tricks our brains to perceive things differently than they actually are. There is no inherit meaning to what we see or hear, it all comes down to context and how we interpret the information. Illusions cheats our way of interpreting the incoming data of our sensory system. This paper will look into our processing of sensory data and explain why a selection of illusions work the way that they do. The illusions that will be used in this paper as examples will focus on optical illusions in relation to color and perspective. It will also look at auditory illusions that uses different aspects of our hearing to work.